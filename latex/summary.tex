\section{Summary}
We're at the end of this short course on statistics for particle physics. Hopefully, some of the common terms in HEP that we mentioned in the first lecture are now clear. There are many other statistical concepts which we didn't have time to cover, and you may come across in your Ph.D,
\begin{itemize}
    \item Unfolding -- There are many publications on the issues of unfolding, removing detector effects to calculate cross-sections at particle level. 
    \item Information theory -- We touched on this when discussing Wilks' theorem (its related to the second derivative of the log-likelihood), but this is a whole topic unto itself. 
    \item Asymptotic theories -- Wilks' theorem is extremely powerful for determining intervals based on likelihood ratio test statistics, however, there are many other test statistics out there with well understood asymptotic behaviour. These may become more common as our computing gets more efficient. 
    \item Goodness of fits -- We covered a common goodness of fit test using likelihood ratios. There are many different tests that are sensitive to different ``features'' in the data. A good practise is to try several of them when checking the consistency of your models with the data. 
    \item Estimators -- We covered maximum likelihood estimators but there are others on the market, some more useful than others or have well known associated intervals.  
    \item Systematic uncertainties -- We only covered how to deal with systematic uncertainties as nuisance parameters, but estimating and dealing with systematic uncertainties in hypothesis testing is a huge field. Furthermore, we didn't cover the case where a parameter in the model isn't specified under one or more hypotheses. This can lead to something known as the \emph{Look elsewhere effect} and there are many interesting methods for dealing with this. 
\end{itemize}
There are several packages used by HEP collaborations that bundle common routines together and help make model building easier, 
\begin{itemize}
    \item \href{https://cds.cern.ch/record/1456844/files/CERN-OPEN-2012-016.pdf}{\textsf{HistFactory}} is used by the ATLAS collaboration, and comes with an XML based model builder for histogram based models. It is based on the \textsf{RooStats} libraries. There is a (almost complete) python based version too \href{https://scikit-hep.org/pyhf/}{\textsf{pyHF}}.
    \item \href{http://cms-analysis.github.io/HiggsAnalysis-CombinedLimit/}{\textsf{Combine}} is \emph{the} standard tool for statistical analysis for the CMS collaboration, using a text-based model definition. It is also based on \textsf{RooFit}, but includes several additional features and optimisations.
    \item \href{https://github.com/zfit/zfit}{\textsf{Zfit}} is a python based framework for building unbinned likelihoods, frequently used by the LHCb collaboration. 
\end{itemize}


Finally, the following is a list of recommended further reading to find out about some of these subjects and other issues in statistics pertinent for particle physics. 
\begin{itemize}
    \item L. Lyons, N. Wardle, ``\emph{Statistical issues in searches for new phenomena in High Energy Physics}'', Journal of Physics G: Nuclear and Particle Physics, Volume 45, Number 3. 
    \item  G. Cowan, ``Statistics'' (section 39) in ``\emph{Review of particle physics}'', Chin. Phys. C 40, 100001 (2016).
    \item O. Behnke, K. Kroninger, G. Schott, T.  Schorner-Sadenius, ``\emph{Data Analysis in High Energy Physics: A Practical Guide to Statistical Methods}'', ISBN: 978-3-527-41058-3 (2013).
    \item K. Cramner, ``\emph{Practical Statistics for the LHC}'', Proceedings, 2011 European School of High-Energy Physics, (2011).
    \item G. J. Feldman, R. D. Cousins, ``\emph{A Unified approach to the classical statistical analysis of small signals}'', Phys. Rev. D57 (1998).
    \item G. Cowan, ``\emph{Statistical Data Analysis}'', ISBN: 978-0-198-50155-8 (1998).
    \item L. Lista, ``\emph{Statistical Methods for Data Analysis in Particle Physics}'', ISBN 978-3-319-20176-4, (2015).
    \item F. James, ``\emph{Statistical Methods in Experimental Physics}'', ISBN: 978-9-812-70527-3 (2006). 
    \item A. Stuart, K. Ord, S. Arnold, ``\emph{Kendall's Advanced theory of Statistics}'', Vol 2A: Classical inference and the linear model, ISBN: 978-0-470-68924-0 (2010). 
\end{itemize}